% Options for packages loaded elsewhere
\PassOptionsToPackage{unicode}{hyperref}
\PassOptionsToPackage{hyphens}{url}
%
\documentclass[
]{article}
\usepackage{amsmath,amssymb}
\usepackage{iftex}
\ifPDFTeX
  \usepackage[T1]{fontenc}
  \usepackage[utf8]{inputenc}
  \usepackage{textcomp} % provide euro and other symbols
\else % if luatex or xetex
  \usepackage{unicode-math} % this also loads fontspec
  \defaultfontfeatures{Scale=MatchLowercase}
  \defaultfontfeatures[\rmfamily]{Ligatures=TeX,Scale=1}
\fi
\usepackage{lmodern}
\ifPDFTeX\else
  % xetex/luatex font selection
\fi
% Use upquote if available, for straight quotes in verbatim environments
\IfFileExists{upquote.sty}{\usepackage{upquote}}{}
\IfFileExists{microtype.sty}{% use microtype if available
  \usepackage[]{microtype}
  \UseMicrotypeSet[protrusion]{basicmath} % disable protrusion for tt fonts
}{}
\makeatletter
\@ifundefined{KOMAClassName}{% if non-KOMA class
  \IfFileExists{parskip.sty}{%
    \usepackage{parskip}
  }{% else
    \setlength{\parindent}{0pt}
    \setlength{\parskip}{6pt plus 2pt minus 1pt}}
}{% if KOMA class
  \KOMAoptions{parskip=half}}
\makeatother
\usepackage{xcolor}
\usepackage[margin=1in]{geometry}
\usepackage{color}
\usepackage{fancyvrb}
\newcommand{\VerbBar}{|}
\newcommand{\VERB}{\Verb[commandchars=\\\{\}]}
\DefineVerbatimEnvironment{Highlighting}{Verbatim}{commandchars=\\\{\}}
% Add ',fontsize=\small' for more characters per line
\usepackage{framed}
\definecolor{shadecolor}{RGB}{248,248,248}
\newenvironment{Shaded}{\begin{snugshade}}{\end{snugshade}}
\newcommand{\AlertTok}[1]{\textcolor[rgb]{0.94,0.16,0.16}{#1}}
\newcommand{\AnnotationTok}[1]{\textcolor[rgb]{0.56,0.35,0.01}{\textbf{\textit{#1}}}}
\newcommand{\AttributeTok}[1]{\textcolor[rgb]{0.13,0.29,0.53}{#1}}
\newcommand{\BaseNTok}[1]{\textcolor[rgb]{0.00,0.00,0.81}{#1}}
\newcommand{\BuiltInTok}[1]{#1}
\newcommand{\CharTok}[1]{\textcolor[rgb]{0.31,0.60,0.02}{#1}}
\newcommand{\CommentTok}[1]{\textcolor[rgb]{0.56,0.35,0.01}{\textit{#1}}}
\newcommand{\CommentVarTok}[1]{\textcolor[rgb]{0.56,0.35,0.01}{\textbf{\textit{#1}}}}
\newcommand{\ConstantTok}[1]{\textcolor[rgb]{0.56,0.35,0.01}{#1}}
\newcommand{\ControlFlowTok}[1]{\textcolor[rgb]{0.13,0.29,0.53}{\textbf{#1}}}
\newcommand{\DataTypeTok}[1]{\textcolor[rgb]{0.13,0.29,0.53}{#1}}
\newcommand{\DecValTok}[1]{\textcolor[rgb]{0.00,0.00,0.81}{#1}}
\newcommand{\DocumentationTok}[1]{\textcolor[rgb]{0.56,0.35,0.01}{\textbf{\textit{#1}}}}
\newcommand{\ErrorTok}[1]{\textcolor[rgb]{0.64,0.00,0.00}{\textbf{#1}}}
\newcommand{\ExtensionTok}[1]{#1}
\newcommand{\FloatTok}[1]{\textcolor[rgb]{0.00,0.00,0.81}{#1}}
\newcommand{\FunctionTok}[1]{\textcolor[rgb]{0.13,0.29,0.53}{\textbf{#1}}}
\newcommand{\ImportTok}[1]{#1}
\newcommand{\InformationTok}[1]{\textcolor[rgb]{0.56,0.35,0.01}{\textbf{\textit{#1}}}}
\newcommand{\KeywordTok}[1]{\textcolor[rgb]{0.13,0.29,0.53}{\textbf{#1}}}
\newcommand{\NormalTok}[1]{#1}
\newcommand{\OperatorTok}[1]{\textcolor[rgb]{0.81,0.36,0.00}{\textbf{#1}}}
\newcommand{\OtherTok}[1]{\textcolor[rgb]{0.56,0.35,0.01}{#1}}
\newcommand{\PreprocessorTok}[1]{\textcolor[rgb]{0.56,0.35,0.01}{\textit{#1}}}
\newcommand{\RegionMarkerTok}[1]{#1}
\newcommand{\SpecialCharTok}[1]{\textcolor[rgb]{0.81,0.36,0.00}{\textbf{#1}}}
\newcommand{\SpecialStringTok}[1]{\textcolor[rgb]{0.31,0.60,0.02}{#1}}
\newcommand{\StringTok}[1]{\textcolor[rgb]{0.31,0.60,0.02}{#1}}
\newcommand{\VariableTok}[1]{\textcolor[rgb]{0.00,0.00,0.00}{#1}}
\newcommand{\VerbatimStringTok}[1]{\textcolor[rgb]{0.31,0.60,0.02}{#1}}
\newcommand{\WarningTok}[1]{\textcolor[rgb]{0.56,0.35,0.01}{\textbf{\textit{#1}}}}
\usepackage{graphicx}
\makeatletter
\def\maxwidth{\ifdim\Gin@nat@width>\linewidth\linewidth\else\Gin@nat@width\fi}
\def\maxheight{\ifdim\Gin@nat@height>\textheight\textheight\else\Gin@nat@height\fi}
\makeatother
% Scale images if necessary, so that they will not overflow the page
% margins by default, and it is still possible to overwrite the defaults
% using explicit options in \includegraphics[width, height, ...]{}
\setkeys{Gin}{width=\maxwidth,height=\maxheight,keepaspectratio}
% Set default figure placement to htbp
\makeatletter
\def\fps@figure{htbp}
\makeatother
\setlength{\emergencystretch}{3em} % prevent overfull lines
\providecommand{\tightlist}{%
  \setlength{\itemsep}{0pt}\setlength{\parskip}{0pt}}
\setcounter{secnumdepth}{-\maxdimen} % remove section numbering
\usepackage{booktabs}
\usepackage{tikz}
\usetikzlibrary{fit}
\usepackage{colortbl}
\ifLuaTeX
  \usepackage{selnolig}  % disable illegal ligatures
\fi
\IfFileExists{bookmark.sty}{\usepackage{bookmark}}{\usepackage{hyperref}}
\IfFileExists{xurl.sty}{\usepackage{xurl}}{} % add URL line breaks if available
\urlstyle{same}
\hypersetup{
  pdftitle={Recitation 8: Time Series and Big Data},
  pdfauthor={Matthew Davis},
  hidelinks,
  pdfcreator={LaTeX via pandoc}}

\title{Recitation 8: Time Series and Big Data}
\author{Matthew Davis}
\date{July 18, 2023}

\begin{document}
\maketitle

\hypertarget{practice-problem-1-the-st.-louis-model}{%
\section{Practice Problem 1: The St.~Louis
Model}\label{practice-problem-1-the-st.-louis-model}}

A model that attracted quite a bit of interest in macroeconomics in the
1970s was the St.~Louis model. The underlying idea was to calculate
fiscal and monetary policy impact and long-run \emph{cumulative dynamic
multipliers}, by relating real output (growth) to real government
expenditure (growth) and real money supply (growth). The assumption was
that both government expenditures and the money supply were exogenous.

In order to investigate the effect of a fiscal and monetary policies on
output, you want to estimate a St.~Louis type model using your quarterly
data (i.e., make sure to use HAC standard errors) and report your
results.

\hypertarget{a-and-b-download-fred-data}{%
\subsection{(a) and (b) Download FRED
Data}\label{a-and-b-download-fred-data}}

\hypertarget{visit-the-federal-reserve-bank-of-st.-louis-where-you-have-access-to-the-fred-and-download-the-data-for-the-required-three-variables-i.e.-real-gdp-gdpc1-real-money-supply-m2real-and-real-government-expenditure-gcec1.}{%
\paragraph{Visit the Federal Reserve Bank of St.~Louis where you have
access to the FRED and download the data for the required three
variables (i.e., real GDP (GDPC1), real money supply (M2REAL), and real
government expenditure
(GCEC1)).}\label{visit-the-federal-reserve-bank-of-st.-louis-where-you-have-access-to-the-fred-and-download-the-data-for-the-required-three-variables-i.e.-real-gdp-gdpc1-real-money-supply-m2real-and-real-government-expenditure-gcec1.}}

Don't need to learn this, but a convenient way of doing this is using
the Quandl package to download the data directly

\begin{Shaded}
\begin{Highlighting}[]
\FunctionTok{Quandl.api\_key}\NormalTok{(}\StringTok{\textquotesingle{}xHu2y3xExQ6bGkGqcYEi\textquotesingle{}}\NormalTok{) }\CommentTok{\# This is my personal API key}
\NormalTok{gdp }\OtherTok{\textless{}{-}} \FunctionTok{Quandl}\NormalTok{(}\StringTok{\textquotesingle{}FRED/GDPC1\textquotesingle{}}\NormalTok{)}
\NormalTok{m2 }\OtherTok{\textless{}{-}} \FunctionTok{Quandl}\NormalTok{(}\StringTok{\textquotesingle{}FRED/M2REAL\textquotesingle{}}\NormalTok{)}
\NormalTok{gov }\OtherTok{\textless{}{-}} \FunctionTok{Quandl}\NormalTok{(}\StringTok{\textquotesingle{}FRED/GCEC1\textquotesingle{}}\NormalTok{)}

\FunctionTok{head}\NormalTok{(gdp)}
\end{Highlighting}
\end{Shaded}

\begin{verbatim}
##         Date    Value
## 1 2021-10-01 19805.96
## 2 2021-07-01 19478.89
## 3 2021-04-01 19368.31
## 4 2021-01-01 19055.65
## 5 2020-10-01 18767.78
## 6 2020-07-01 18560.77
\end{verbatim}

\begin{Shaded}
\begin{Highlighting}[]
\FunctionTok{head}\NormalTok{(m2)}
\end{Highlighting}
\end{Shaded}

\begin{verbatim}
##         Date  Value
## 1 2021-12-01 7724.4
## 2 2021-11-01 7696.6
## 3 2021-10-01 7660.0
## 4 2021-09-01 7657.1
## 5 2021-08-01 7629.0
## 6 2021-07-01 7560.6
\end{verbatim}

\begin{Shaded}
\begin{Highlighting}[]
\FunctionTok{head}\NormalTok{(gov)}
\end{Highlighting}
\end{Shaded}

\begin{verbatim}
##         Date    Value
## 1 2021-10-01 3356.829
## 2 2021-07-01 3381.574
## 3 2021-04-01 3373.765
## 4 2021-01-01 3390.921
## 5 2020-10-01 3356.030
## 6 2020-07-01 3360.238
\end{verbatim}

\hypertarget{the-real-money-supply-m2-is-available-on-a-monthly-frequency-basis-dont-forget-to-convert-it-into-a-quarterly-frequency-variable-to-match-it-with-the-other-two-variables.-hint-is-m2-a-stock-or-flow-variable-although-you-may-take-the-last-month-of-each-quarter-or-the-three-month-average-as-your-quarterly-value-here-use-the-first-month-of-each-quarter.}{%
\paragraph{\texorpdfstring{The real money supply \emph{m2} is available
on a monthly frequency basis: don't forget to convert it into a
quarterly frequency variable to match it with the other two variables.
Hint: Is \emph{m2} a stock or flow variable? Although, you may take the
last month of each quarter or the three-month average as your quarterly
value, here use the first month of each
quarter.}{The real money supply m2 is available on a monthly frequency basis: don't forget to convert it into a quarterly frequency variable to match it with the other two variables. Hint: Is m2 a stock or flow variable? Although, you may take the last month of each quarter or the three-month average as your quarterly value, here use the first month of each quarter.}}\label{the-real-money-supply-m2-is-available-on-a-monthly-frequency-basis-dont-forget-to-convert-it-into-a-quarterly-frequency-variable-to-match-it-with-the-other-two-variables.-hint-is-m2-a-stock-or-flow-variable-although-you-may-take-the-last-month-of-each-quarter-or-the-three-month-average-as-your-quarterly-value-here-use-the-first-month-of-each-quarter.}}

Since M2 is a stock variable, you can take the value of this series
corresponding to the first month of each quarter to get the quarterly
values of this variable.

\begin{Shaded}
\begin{Highlighting}[]
\NormalTok{fred }\OtherTok{\textless{}{-}} \FunctionTok{left\_join}\NormalTok{(gdp, m2, }\AttributeTok{by =} \StringTok{\textquotesingle{}Date\textquotesingle{}}\NormalTok{) }\SpecialCharTok{\%\textgreater{}\%}
  \FunctionTok{left\_join}\NormalTok{(gov, }\AttributeTok{by =} \StringTok{\textquotesingle{}Date\textquotesingle{}}\NormalTok{)}
\FunctionTok{head}\NormalTok{(fred)}
\end{Highlighting}
\end{Shaded}

\begin{verbatim}
##         Date  Value.x Value.y    Value
## 1 2021-10-01 19805.96  7660.0 3356.829
## 2 2021-07-01 19478.89  7560.6 3381.574
## 3 2021-04-01 19368.31  7550.3 3373.765
## 4 2021-01-01 19055.65  7396.4 3390.921
## 5 2020-10-01 18767.78  7201.0 3356.030
## 6 2020-07-01 18560.77  7084.5 3360.238
\end{verbatim}

\begin{Shaded}
\begin{Highlighting}[]
\NormalTok{fred }\SpecialCharTok{\%\textless{}\textgreater{}\%} \FunctionTok{rename}\NormalTok{(}\AttributeTok{gdp =}\NormalTok{ Value.x,}
                 \AttributeTok{m2 =}\NormalTok{ Value.y,}
                 \AttributeTok{gov =}\NormalTok{ Value)}
\FunctionTok{head}\NormalTok{(fred)}
\end{Highlighting}
\end{Shaded}

\begin{verbatim}
##         Date      gdp     m2      gov
## 1 2021-10-01 19805.96 7660.0 3356.829
## 2 2021-07-01 19478.89 7560.6 3381.574
## 3 2021-04-01 19368.31 7550.3 3373.765
## 4 2021-01-01 19055.65 7396.4 3390.921
## 5 2020-10-01 18767.78 7201.0 3356.030
## 6 2020-07-01 18560.77 7084.5 3360.238
\end{verbatim}

Notice that our time variable, Date, is not just a number variable but a
`Date' variable.

\hypertarget{the-sample-period-should-be-from-first-quarter-of-1960-to-the-fourth-quarter-of-2019-i.e.-1960q1-to-2019q4.}{%
\paragraph{The sample period should be from first quarter of 1960 to the
fourth quarter of 2019 (i.e., 1960q1 to
2019q4).}\label{the-sample-period-should-be-from-first-quarter-of-1960-to-the-fourth-quarter-of-2019-i.e.-1960q1-to-2019q4.}}

Quarters in this dataset are defined by the first day of the first month
in that quarter, i.e.~the 1st of January, April, July, and October

\begin{Shaded}
\begin{Highlighting}[]
\NormalTok{fred }\SpecialCharTok{\%\textless{}\textgreater{}\%} \FunctionTok{filter}\NormalTok{(Date }\SpecialCharTok{\textgreater{}=} \StringTok{\textquotesingle{}1960{-}01{-}01\textquotesingle{}}\NormalTok{) }\SpecialCharTok{\%\textgreater{}\%}
  \FunctionTok{filter}\NormalTok{(Date }\SpecialCharTok{\textless{}=} \StringTok{\textquotesingle{}2019{-}12{-}31\textquotesingle{}}\NormalTok{)}
\end{Highlighting}
\end{Shaded}

As a sanity check, let's plot one of these variables as a line graph:

\begin{Shaded}
\begin{Highlighting}[]
\FunctionTok{ggplot}\NormalTok{(fred, }\FunctionTok{aes}\NormalTok{(}\AttributeTok{x =}\NormalTok{ Date, }\AttributeTok{y =}\NormalTok{ m2)) }\SpecialCharTok{+}
  \FunctionTok{theme\_bw}\NormalTok{() }\SpecialCharTok{+}
  \FunctionTok{geom\_line}\NormalTok{() }\SpecialCharTok{+}
  \FunctionTok{xlab}\NormalTok{(}\StringTok{\textquotesingle{}Year\textquotesingle{}}\NormalTok{) }\SpecialCharTok{+}
  \FunctionTok{ylab}\NormalTok{(}\StringTok{\textquotesingle{}\textquotesingle{}}\NormalTok{) }\SpecialCharTok{+}
  \FunctionTok{ggtitle}\NormalTok{(}\StringTok{\textquotesingle{}Time series of M2 money supply, quarterly\textquotesingle{}}\NormalTok{)}
\end{Highlighting}
\end{Shaded}

\includegraphics{08-TimeSeries-DynamicCausalEffects_files/figure-latex/unnamed-chunk-5-1.pdf}

We can also derive the autocorrelations for this series either as a plot
of a corellogram\ldots{}

\begin{Shaded}
\begin{Highlighting}[]
\FunctionTok{acf}\NormalTok{(fred}\SpecialCharTok{$}\NormalTok{m2,}
    \AttributeTok{na.action =}\NormalTok{ na.pass) }\CommentTok{\# ignore any missing data/gaps in the time series}
\end{Highlighting}
\end{Shaded}

\includegraphics{08-TimeSeries-DynamicCausalEffects_files/figure-latex/unnamed-chunk-6-1.pdf}

Or by displaying the estimated autocorrelations, say up to 15 lags:

\begin{Shaded}
\begin{Highlighting}[]
\FunctionTok{acf}\NormalTok{(fred}\SpecialCharTok{$}\NormalTok{m2,}
    \AttributeTok{na.action =}\NormalTok{ na.exclude, }\CommentTok{\# Ignore any missing data}
    \AttributeTok{lag.max =} \DecValTok{15}\NormalTok{, }\CommentTok{\# Number of lags to include}
    \AttributeTok{plot =} \ConstantTok{FALSE}\NormalTok{)}
\end{Highlighting}
\end{Shaded}

\begin{verbatim}
## 
## Autocorrelations of series 'fred$m2', by lag
## 
##     0     1     2     3     4     5     6     7     8     9    10    11    12 
## 1.000 0.981 0.962 0.944 0.925 0.906 0.887 0.868 0.849 0.830 0.810 0.791 0.771 
##    13    14    15 
## 0.752 0.734 0.715
\end{verbatim}

\hypertarget{compute-the-growth-rates-of-these-three-variables-after-you-first-transform-them-into-natural-logarithm-and-namelabel-them-ygrowth-mgrowth-and-ggrowth}{%
\paragraph{\texorpdfstring{Compute the growth rates of these three
variables after you first transform them into natural logarithm and
name/label them \emph{ygrowth}, \emph{mgrowth}, and
\emph{ggrowth}}{Compute the growth rates of these three variables after you first transform them into natural logarithm and name/label them ygrowth, mgrowth, and ggrowth}}\label{compute-the-growth-rates-of-these-three-variables-after-you-first-transform-them-into-natural-logarithm-and-namelabel-them-ygrowth-mgrowth-and-ggrowth}}

For a variable \(X_t\), we'll compute the growth rate \(x_t\) using the
following transformation:

\[
x_t = \log(X_t)-\log(X_{t-1})
\]

\begin{Shaded}
\begin{Highlighting}[]
\NormalTok{fred }\SpecialCharTok{\%\textless{}\textgreater{}\%} \FunctionTok{mutate}\NormalTok{(}\AttributeTok{ygrowth =} \FunctionTok{log}\NormalTok{(gdp)}\SpecialCharTok{{-}}\FunctionTok{lag}\NormalTok{(}\FunctionTok{log}\NormalTok{(gdp)),}
                 \AttributeTok{mgrowth =} \FunctionTok{log}\NormalTok{(m2)}\SpecialCharTok{{-}}\FunctionTok{lag}\NormalTok{(}\FunctionTok{log}\NormalTok{(m2)),}
                 \AttributeTok{ggrowth =} \FunctionTok{log}\NormalTok{(gov)}\SpecialCharTok{{-}}\FunctionTok{lag}\NormalTok{(}\FunctionTok{log}\NormalTok{(gov)))}
\end{Highlighting}
\end{Shaded}

\hypertarget{c-estimate-a-distributed-lag-model-of-ygrowth-on-current-period-mgrowth-the-effect-of-a-monetary-policy-on-current-quarters-output-growth}{%
\subsection{\texorpdfstring{(c) Estimate a distributed lag model of
\emph{ygrowth} on current-period \emph{mgrowth}, the effect of a
monetary policy on current quarter's output
growth}{(c) Estimate a distributed lag model of ygrowth on current-period mgrowth, the effect of a monetary policy on current quarter's output growth}}\label{c-estimate-a-distributed-lag-model-of-ygrowth-on-current-period-mgrowth-the-effect-of-a-monetary-policy-on-current-quarters-output-growth}}

\hypertarget{d-estimate-a-distributed-lag-model-of-ygrowth-on-current-period-ggrowth-the-effect-of-a-fiscal-policy-on-current-quarters-output-growth}{%
\subsection{\texorpdfstring{(d) Estimate a distributed lag model of
\emph{ygrowth} on current-period \emph{ggrowth}, the effect of a fiscal
policy on current quarter's output
growth}{(d) Estimate a distributed lag model of ygrowth on current-period ggrowth, the effect of a fiscal policy on current quarter's output growth}}\label{d-estimate-a-distributed-lag-model-of-ygrowth-on-current-period-ggrowth-the-effect-of-a-fiscal-policy-on-current-quarters-output-growth}}

We'll use our familiar feols package to estimate a series of distributed
lag models. We'll find it convenient to use the same panel methods for
time series, allowing us to use mostly the same commands we've used
before. This will require us to define our unit and and time variables
(since fixest is intended for panel datasets). Since time series are
essentially just panel datasets with \(T\) time periods but only \(N=1\)
units, let's just create a unit variable that takes on an arbitrary
value for all observations:

\begin{Shaded}
\begin{Highlighting}[]
\NormalTok{fred }\SpecialCharTok{\%\textless{}\textgreater{}\%} \FunctionTok{mutate}\NormalTok{(}\AttributeTok{unit =} \StringTok{\textquotesingle{}same\textquotesingle{}}\NormalTok{)}
\end{Highlighting}
\end{Shaded}

This will also be necessary to make use of the other new component: new
standard errors for time series and panel datasets which are robust to
both heteroskedasticity and autocorrelation (HAC). We call these
Newey-West (NW) standard errors and they are convenientliy calculated
using the \(NW(m)\) function.

These standard errors require us to determine the appropriate lag
truncation parameter \(m\). This is a matter of applying the
rule-of-thumb formula from the textbook and rounding up:

\begin{Shaded}
\begin{Highlighting}[]
\NormalTok{m }\OtherTok{\textless{}{-}} \FloatTok{0.75}\SpecialCharTok{*}\FunctionTok{nrow}\NormalTok{(fred)}\SpecialCharTok{\^{}}\NormalTok{(}\DecValTok{1}\SpecialCharTok{/}\DecValTok{3}\NormalTok{)}
\NormalTok{m}
\end{Highlighting}
\end{Shaded}

\begin{verbatim}
## [1] 4.660849
\end{verbatim}

\begin{Shaded}
\begin{Highlighting}[]
\CommentTok{\# Round up}
\NormalTok{m }\OtherTok{\textless{}{-}} \FunctionTok{ceiling}\NormalTok{(m)}
\NormalTok{m}
\end{Highlighting}
\end{Shaded}

\begin{verbatim}
## [1] 5
\end{verbatim}

Last note: we can of course use the same functions to estimate
autoregressive models AR(p). In this case, we do not want to use HAC
standard errors and in addition, we'll want to specify iid standard
errors so we don't cluster standard errors

Now we can estimate the desired models:

\begin{Shaded}
\begin{Highlighting}[]
\NormalTok{mod.c }\OtherTok{\textless{}{-}} \FunctionTok{feols}\NormalTok{(ygrowth }\SpecialCharTok{\textasciitilde{}}\NormalTok{ mgrowth, fred,}
               \AttributeTok{panel.id =} \FunctionTok{c}\NormalTok{(}\StringTok{\textquotesingle{}unit\textquotesingle{}}\NormalTok{, }\StringTok{\textquotesingle{}Date\textquotesingle{}}\NormalTok{),}
               \AttributeTok{vcov =} \FunctionTok{NW}\NormalTok{(}\DecValTok{5}\NormalTok{))}
\end{Highlighting}
\end{Shaded}

\begin{verbatim}
## NOTE: 1 observation removed because of NA values (LHS: 1, RHS: 1).
\end{verbatim}

\begin{Shaded}
\begin{Highlighting}[]
\NormalTok{mod.d }\OtherTok{\textless{}{-}} \FunctionTok{feols}\NormalTok{(ygrowth }\SpecialCharTok{\textasciitilde{}}\NormalTok{ ggrowth, fred,}
               \AttributeTok{panel.id =} \FunctionTok{c}\NormalTok{(}\StringTok{\textquotesingle{}unit\textquotesingle{}}\NormalTok{, }\StringTok{\textquotesingle{}Date\textquotesingle{}}\NormalTok{),}
               \AttributeTok{vcov =} \FunctionTok{NW}\NormalTok{(}\DecValTok{5}\NormalTok{))}
\end{Highlighting}
\end{Shaded}

\begin{verbatim}
## NOTE: 1 observation removed because of NA values (LHS: 1, RHS: 1).
\end{verbatim}

\begin{Shaded}
\begin{Highlighting}[]
\FunctionTok{etable}\NormalTok{(mod.c, mod.d, }\AttributeTok{markdown =}\NormalTok{ T)}
\end{Highlighting}
\end{Shaded}

\begingroup
\centering
\begin{tabular}{lcc}
   \tabularnewline \midrule \midrule
   Dependent Variable: & \multicolumn{2}{c}{ygrowth}\\
   Model:         & (1)             & (2)\\  
   \midrule
   \emph{Variables}\\
   Constant       & -0.0066$^{***}$ & -0.0064$^{***}$\\   
                  & (0.0008)        & (0.0007)\\   
   mgrowth        & 0.1135          &   \\   
                  & (0.0781)        &   \\   
   ggrowth        &                 & 0.2148$^{***}$\\   
                  &                 & (0.0520)\\   
   \midrule
   \emph{Fit statistics}\\
   Observations   & 239             & 239\\  
   R$^2$          & 0.02722         & 0.06508\\  
   Adjusted R$^2$ & 0.02312         & 0.06114\\  
   \midrule \midrule
   \multicolumn{3}{l}{\emph{Newey-West (L=5) standard-errors in parentheses}}\\
   \multicolumn{3}{l}{\emph{Signif. Codes: ***: 0.01, **: 0.05, *: 0.1}}\\
\end{tabular}
\par\endgroup

\hypertarget{e-estimate-a-distributed-lag-model-of-ygrowth-on-current-and-next-quarters-monetary-policy-on-output-growth}{%
\subsection{\texorpdfstring{(e) Estimate a distributed lag model of
\emph{ygrowth} on current and next quarter's monetary policy on output
growth}{(e) Estimate a distributed lag model of ygrowth on current and next quarter's monetary policy on output growth}}\label{e-estimate-a-distributed-lag-model-of-ygrowth-on-current-and-next-quarters-monetary-policy-on-output-growth}}

\hypertarget{f-estimate-a-distributed-lag-model-of-ygrowth-on-current-and-next-quarters-fiscal-policy}{%
\subsection{\texorpdfstring{(f) Estimate a distributed lag model of
\emph{ygrowth} on current and next quarter's fiscal
policy}{(f) Estimate a distributed lag model of ygrowth on current and next quarter's fiscal policy}}\label{f-estimate-a-distributed-lag-model-of-ygrowth-on-current-and-next-quarters-fiscal-policy}}

To clarify the ambiguous language of the question, we are being asked to
regress GDP growth between period t and t+1 to changes to
monetary/fiscal policy between periods t and t+1 and changes to
monetary/fiscal policy between periods t-1 and t.

Other than the standard errors, the other new component here is that our
models are ``dynamic'', meaning they relate observations from different
time periods to one another. Here, we'll find it very convenient to make
use of the fixest package, which has functions l(), d(), and f(), which
allow us to refer to lagged, differenced, and lead terms in a regression
formula without having to manually create new variables.

Estimating the above models:

\begin{Shaded}
\begin{Highlighting}[]
\NormalTok{mod.e }\OtherTok{\textless{}{-}} \FunctionTok{feols}\NormalTok{(ygrowth }\SpecialCharTok{\textasciitilde{}} \FunctionTok{l}\NormalTok{(mgrowth, }\DecValTok{0}\SpecialCharTok{:}\DecValTok{1}\NormalTok{), fred,}
               \AttributeTok{panel.id =} \FunctionTok{c}\NormalTok{(}\StringTok{\textquotesingle{}unit\textquotesingle{}}\NormalTok{, }\StringTok{\textquotesingle{}Date\textquotesingle{}}\NormalTok{),}
               \AttributeTok{vcov =} \FunctionTok{NW}\NormalTok{(}\DecValTok{5}\NormalTok{))}
\end{Highlighting}
\end{Shaded}

\begin{verbatim}
## NOTE: 2 observations removed because of NA values (LHS: 1, RHS: 2).
\end{verbatim}

\begin{Shaded}
\begin{Highlighting}[]
\NormalTok{mod.f }\OtherTok{\textless{}{-}} \FunctionTok{feols}\NormalTok{(ygrowth }\SpecialCharTok{\textasciitilde{}} \FunctionTok{l}\NormalTok{(ggrowth, }\DecValTok{0}\SpecialCharTok{:}\DecValTok{1}\NormalTok{),}
\NormalTok{               fred,}\AttributeTok{panel.id =} \FunctionTok{c}\NormalTok{(}\StringTok{\textquotesingle{}unit\textquotesingle{}}\NormalTok{, }\StringTok{\textquotesingle{}Date\textquotesingle{}}\NormalTok{),}
               \AttributeTok{vcov =} \FunctionTok{NW}\NormalTok{(}\DecValTok{5}\NormalTok{))}
\end{Highlighting}
\end{Shaded}

\begin{verbatim}
## NOTE: 2 observations removed because of NA values (LHS: 1, RHS: 2).
\end{verbatim}

\begin{Shaded}
\begin{Highlighting}[]
\CommentTok{\# Or combining them into one command}
\NormalTok{mods.ef }\OtherTok{\textless{}{-}} \FunctionTok{feols}\NormalTok{(ygrowth }\SpecialCharTok{\textasciitilde{}} \FunctionTok{sw}\NormalTok{(}\FunctionTok{l}\NormalTok{(mgrowth, }\DecValTok{0}\SpecialCharTok{:}\DecValTok{1}\NormalTok{),}
                                 \FunctionTok{l}\NormalTok{(ggrowth, }\DecValTok{0}\SpecialCharTok{:}\DecValTok{1}\NormalTok{)),}
\NormalTok{                 fred,}\AttributeTok{panel.id =} \FunctionTok{c}\NormalTok{(}\StringTok{\textquotesingle{}unit\textquotesingle{}}\NormalTok{, }\StringTok{\textquotesingle{}Date\textquotesingle{}}\NormalTok{),}
                 \AttributeTok{vcov =} \FunctionTok{NW}\NormalTok{(}\DecValTok{5}\NormalTok{))}
\end{Highlighting}
\end{Shaded}

\begin{verbatim}
## NOTE: 1 observation removed because of NA values (LHS: 1).
##       |-> this msg only concerns the variables common to all estimations
\end{verbatim}

\begin{Shaded}
\begin{Highlighting}[]
\FunctionTok{etable}\NormalTok{(mods.ef, }\AttributeTok{markdown =}\NormalTok{ T)}
\end{Highlighting}
\end{Shaded}

\begingroup
\centering
\begin{tabular}{lcc}
   \tabularnewline \midrule \midrule
   Dependent Variable: & \multicolumn{2}{c}{ygrowth}\\
   Model:         & (1)             & (2)\\  
   \midrule
   \emph{Variables}\\
   Constant       & -0.0058$^{***}$ & -0.0067$^{***}$\\   
                  & (0.0008)        & (0.0006)\\   
   mgrowth        & 0.0165          &   \\   
                  & (0.0629)        &   \\   
   l(mgrowth,1)   & 0.2069$^{***}$  &   \\   
                  & (0.0650)        &   \\   
   ggrowth        &                 & 0.2313$^{***}$\\   
                  &                 & (0.0512)\\   
   l(ggrowth,1)   &                 & -0.0622\\   
                  &                 & (0.0454)\\   
   \midrule
   \emph{Fit statistics}\\
   Observations   & 238             & 238\\  
   R$^2$          & 0.09842         & 0.07352\\  
   Adjusted R$^2$ & 0.09075         & 0.06563\\  
   \midrule \midrule
   \multicolumn{3}{l}{\emph{Newey-West (L=5) standard-errors in parentheses}}\\
   \multicolumn{3}{l}{\emph{Signif. Codes: ***: 0.01, **: 0.05, *: 0.1}}\\
\end{tabular}
\par\endgroup

Notice the second argument in the lag function l() can take on a vector
of numbers if we want to include multiple lags (including lag 0 here).

\hypertarget{what-is-the-impact-multiplier-explain-the-meaning.}{%
\paragraph{What is the impact multiplier? Explain the
meaning.}\label{what-is-the-impact-multiplier-explain-the-meaning.}}

\hypertarget{what-is-the-cumulative-multiplier-explain-the-meaning.}{%
\paragraph{What is the cumulative multiplier? Explain the
meaning.}\label{what-is-the-cumulative-multiplier-explain-the-meaning.}}

\hypertarget{g-estimate-a-distributed-lag-model-of-ygrowth-on-the-change-i.e.-the-first-difference-in-current-and-four-lags-of-mgrowth-and-ggrowth-to-mimic-the-original-st.-louis-equation}{%
\subsection{\texorpdfstring{(g) Estimate a distributed lag model of
\emph{ygrowth} on the change (i.e., the \emph{first difference}) in
current and four lags of \emph{mgrowth} and \emph{ggrowth} (to mimic the
original St.~Louis
Equation)}{(g) Estimate a distributed lag model of ygrowth on the change (i.e., the first difference) in current and four lags of mgrowth and ggrowth (to mimic the original St.~Louis Equation)}}\label{g-estimate-a-distributed-lag-model-of-ygrowth-on-the-change-i.e.-the-first-difference-in-current-and-four-lags-of-mgrowth-and-ggrowth-to-mimic-the-original-st.-louis-equation}}

\begin{Shaded}
\begin{Highlighting}[]
\CommentTok{\# Create first{-}differenced versions of our regressors}
\NormalTok{fred }\SpecialCharTok{\%\textless{}\textgreater{}\%} \FunctionTok{mutate}\NormalTok{(}\AttributeTok{diff.mgrowth =}\NormalTok{ mgrowth}\SpecialCharTok{{-}}\FunctionTok{lag}\NormalTok{(mgrowth),}
                 \AttributeTok{diff.ggrowth =}\NormalTok{ ggrowth}\SpecialCharTok{{-}}\FunctionTok{lag}\NormalTok{(ggrowth))}

\NormalTok{mod.g }\OtherTok{\textless{}{-}} \FunctionTok{feols}\NormalTok{(ygrowth }\SpecialCharTok{\textasciitilde{}} \FunctionTok{l}\NormalTok{(diff.mgrowth, }\DecValTok{0}\SpecialCharTok{:}\DecValTok{3}\NormalTok{) }\SpecialCharTok{+} \FunctionTok{l}\NormalTok{(mgrowth, }\DecValTok{4}\NormalTok{) }\SpecialCharTok{+} \FunctionTok{l}\NormalTok{(diff.ggrowth, }\DecValTok{0}\SpecialCharTok{:}\DecValTok{3}\NormalTok{) }\SpecialCharTok{+} \FunctionTok{l}\NormalTok{(ggrowth, }\DecValTok{4}\NormalTok{),}
\NormalTok{               fred, }\AttributeTok{panel.id =} \FunctionTok{c}\NormalTok{(}\StringTok{\textquotesingle{}unit\textquotesingle{}}\NormalTok{, }\StringTok{\textquotesingle{}Date\textquotesingle{}}\NormalTok{),}
               \AttributeTok{vcov =} \FunctionTok{NW}\NormalTok{(}\DecValTok{5}\NormalTok{))}
\end{Highlighting}
\end{Shaded}

\begin{verbatim}
## NOTE: 6 observations removed because of NA values (LHS: 1, RHS: 6).
\end{verbatim}

\begin{Shaded}
\begin{Highlighting}[]
\FunctionTok{etable}\NormalTok{(mod.g, }\AttributeTok{markdown =}\NormalTok{ T)}
\end{Highlighting}
\end{Shaded}

\begingroup
\centering
\begin{tabular}{lc}
   \tabularnewline \midrule \midrule
   Dependent Variable: & ygrowth\\  
   Model:              & (1)\\  
   \midrule
   \emph{Variables}\\
   Constant            & -0.0066$^{***}$\\   
                       & (0.0008)\\   
   diff.mgrowth        & 0.1687$^{***}$\\   
                       & (0.0485)\\   
   l(diff.mgrowth,1)   & 0.1505$^{**}$\\   
                       & (0.0591)\\   
   l(diff.mgrowth,2)   & 0.0415\\   
                       & (0.0744)\\   
   l(diff.mgrowth,3)   & -0.0163\\   
                       & (0.0627)\\   
   l(mgrowth,4)        & 0.1120$^{**}$\\   
                       & (0.0461)\\   
   diff.ggrowth        & -0.0158\\   
                       & (0.0524)\\   
   l(diff.ggrowth,1)   & -0.2164$^{***}$\\   
                       & (0.0640)\\   
   l(diff.ggrowth,2)   & -0.1462$^{**}$\\   
                       & (0.0657)\\   
   l(diff.ggrowth,3)   & -0.1472$^{***}$\\   
                       & (0.0488)\\   
   l(ggrowth,4)        & 0.0346\\   
                       & (0.0498)\\   
   \midrule
   \emph{Fit statistics}\\
   Observations        & 234\\  
   R$^2$               & 0.18118\\  
   Adjusted R$^2$      & 0.14446\\  
   \midrule \midrule
   \multicolumn{2}{l}{\emph{Newey-West (L=5) standard-errors in parentheses}}\\
   \multicolumn{2}{l}{\emph{Signif. Codes: ***: 0.01, **: 0.05, *: 0.1}}\\
\end{tabular}
\par\endgroup

\hypertarget{h-assuming-that-money-and-government-expenditures-are-exogenous-what-do-the-coefficients-represent-calculate-the-h-period-cumulative-dynamic-multipliers-from-these.}{%
\subsection{\texorpdfstring{(h) Assuming that money and government
expenditures are exogenous, what do the coefficients represent?
Calculate the \emph{h}-period cumulative dynamic multipliers from
these.}{(h) Assuming that money and government expenditures are exogenous, what do the coefficients represent? Calculate the h-period cumulative dynamic multipliers from these.}}\label{h-assuming-that-money-and-government-expenditures-are-exogenous-what-do-the-coefficients-represent-calculate-the-h-period-cumulative-dynamic-multipliers-from-these.}}

\begin{Shaded}
\begin{Highlighting}[]
\NormalTok{stl.est }\OtherTok{\textless{}{-}} \FunctionTok{coeftable}\NormalTok{(mod.g)}
\NormalTok{stl.est}
\end{Highlighting}
\end{Shaded}

\begin{verbatim}
##                        Estimate   Std. Error    t value     Pr(>|t|)
## (Intercept)        -0.006592834 0.0007511044 -8.7775207 3.636806e-16
## diff.mgrowth        0.168727833 0.0484537481  3.4822452 5.936453e-04
## l(diff.mgrowth, 1)  0.150496282 0.0591134932  2.5458871 1.154554e-02
## l(diff.mgrowth, 2)  0.041516018 0.0744361091  0.5577403 5.775572e-01
## l(diff.mgrowth, 3) -0.016273219 0.0626708169 -0.2596618 7.953539e-01
## l(mgrowth, 4)       0.112017586 0.0461457772  2.4274721 1.596348e-02
## diff.ggrowth       -0.015797496 0.0524098815 -0.3014221 7.633615e-01
## l(diff.ggrowth, 1) -0.216419889 0.0639485787 -3.3842799 8.372077e-04
## l(diff.ggrowth, 2) -0.146205740 0.0657339718 -2.2242037 2.709386e-02
## l(diff.ggrowth, 3) -0.147186933 0.0488436594 -3.0134297 2.868565e-03
## l(ggrowth, 4)       0.034571318 0.0497538378  0.6948473 4.878433e-01
## attr(,"type")
## [1] "Newey-West (L=5)"
\end{verbatim}

Assuming money and government expenditures are exogenous, the
coefficients estimated represent the (monetary/fiscal) cumulative
multipliers. If you differenced them out, they would represent the
dynamic multipliers. The coefficients are easy to plot when estimated
through fixest:

\begin{Shaded}
\begin{Highlighting}[]
\CommentTok{\# Collect the cumulative multipliers}
\NormalTok{cumu }\OtherTok{\textless{}{-}} \FunctionTok{data.frame}\NormalTok{(}\AttributeTok{Estimate =}\NormalTok{ stl.est[}\SpecialCharTok{{-}}\DecValTok{1}\NormalTok{,}\StringTok{\textquotesingle{}Estimate\textquotesingle{}}\NormalTok{]) }\SpecialCharTok{\%\textgreater{}\%}
  \FunctionTok{mutate}\NormalTok{(}\AttributeTok{Lag =} \FunctionTok{rep}\NormalTok{(}\DecValTok{0}\SpecialCharTok{:}\DecValTok{4}\NormalTok{, }\DecValTok{2}\NormalTok{),}
         \AttributeTok{Policy =} \FunctionTok{rep}\NormalTok{(}\FunctionTok{c}\NormalTok{(}\StringTok{\textquotesingle{}Monetary\textquotesingle{}}\NormalTok{, }\StringTok{\textquotesingle{}Fiscal\textquotesingle{}}\NormalTok{), }\AttributeTok{each =} \DecValTok{5}\NormalTok{))}
\NormalTok{dyna }\OtherTok{\textless{}{-}} \FunctionTok{group\_by}\NormalTok{(cumu, Policy) }\SpecialCharTok{\%\textgreater{}\%}
  \FunctionTok{mutate}\NormalTok{(}\AttributeTok{Estimate =} \FunctionTok{ifelse}\NormalTok{(Lag }\SpecialCharTok{==} \DecValTok{0}\NormalTok{, Estimate, Estimate}\SpecialCharTok{{-}}\FunctionTok{lag}\NormalTok{(Estimate)))}

\NormalTok{multipliers }\OtherTok{\textless{}{-}} \FunctionTok{rbind}\NormalTok{(cumu, dyna) }\SpecialCharTok{\%\textgreater{}\%}
  \FunctionTok{mutate}\NormalTok{(}\AttributeTok{Multiplier =} \FunctionTok{rep}\NormalTok{(}\FunctionTok{c}\NormalTok{(}\StringTok{\textquotesingle{}Cumulative\textquotesingle{}}\NormalTok{, }\StringTok{\textquotesingle{}Dynamic\textquotesingle{}}\NormalTok{), }\AttributeTok{each =} \DecValTok{10}\NormalTok{))}

\NormalTok{multi.table }\OtherTok{\textless{}{-}} \FunctionTok{pivot\_wider}\NormalTok{(multipliers,}
                           \AttributeTok{names\_from =} \FunctionTok{c}\NormalTok{(Policy, Multiplier),}
                           \AttributeTok{values\_from =}\NormalTok{ Estimate)}
\FunctionTok{print}\NormalTok{(multi.table)}
\end{Highlighting}
\end{Shaded}

\begin{verbatim}
## # A tibble: 5 x 5
##     Lag Monetary_Cumulative Fiscal_Cumulative Monetary_Dynamic Fiscal_Dynamic
##   <int>               <dbl>             <dbl>            <dbl>          <dbl>
## 1     0              0.169            -0.0158           0.169       -0.0158  
## 2     1              0.150            -0.216           -0.0182      -0.201   
## 3     2              0.0415           -0.146           -0.109        0.0702  
## 4     3             -0.0163           -0.147           -0.0578      -0.000981
## 5     4              0.112             0.0346           0.128        0.182
\end{verbatim}

\hypertarget{how-can-you-test-for-the-statistical-significance-of-the-cumulative-dynamic-multipliers-and-the-long-run-cumulative-dynamic-multiplier}{%
\paragraph{How can you test for the statistical significance of the
cumulative dynamic multipliers and the long-run cumulative dynamic
multiplier?}\label{how-can-you-test-for-the-statistical-significance-of-the-cumulative-dynamic-multipliers-and-the-long-run-cumulative-dynamic-multiplier}}

\hypertarget{i-sketch-the-estimated-dynamic-and-cumulative-dynamic-fiscal-and-monetary-multipliers-similar-to-fig-16.2-in-the-textbook}{%
\subsection{(i) Sketch the estimated dynamic and cumulative dynamic
fiscal and monetary multipliers (similar to Fig 16.2 in the
textbook)}\label{i-sketch-the-estimated-dynamic-and-cumulative-dynamic-fiscal-and-monetary-multipliers-similar-to-fig-16.2-in-the-textbook}}

\begin{Shaded}
\begin{Highlighting}[]
\FunctionTok{ggplot}\NormalTok{(multipliers, }\FunctionTok{aes}\NormalTok{(}\AttributeTok{x =}\NormalTok{ Lag, }\AttributeTok{y =}\NormalTok{ Estimate, }\AttributeTok{color =}\NormalTok{ Multiplier)) }\SpecialCharTok{+}
  \FunctionTok{theme\_bw}\NormalTok{() }\SpecialCharTok{+}
  \FunctionTok{geom\_line}\NormalTok{() }\SpecialCharTok{+}
  \FunctionTok{theme}\NormalTok{(}\AttributeTok{legend.position =} \StringTok{\textquotesingle{}top\textquotesingle{}}\NormalTok{) }\SpecialCharTok{+}
  \FunctionTok{ggtitle}\NormalTok{(}\StringTok{\textquotesingle{}Estimated multipliers\textquotesingle{}}\NormalTok{) }\SpecialCharTok{+}
  \FunctionTok{xlab}\NormalTok{(}\StringTok{\textquotesingle{}Lags\textquotesingle{}}\NormalTok{) }\SpecialCharTok{+} \FunctionTok{ylab}\NormalTok{(}\StringTok{\textquotesingle{}\textquotesingle{}}\NormalTok{) }\SpecialCharTok{+} \FunctionTok{facet\_grid}\NormalTok{(}\SpecialCharTok{\textasciitilde{}}\NormalTok{ Policy)}
\end{Highlighting}
\end{Shaded}

\includegraphics{08-TimeSeries-DynamicCausalEffects_files/figure-latex/unnamed-chunk-16-1.pdf}

\hypertarget{j-for-these-coefficients-to-represent-dynamic-ultipliers-the-money-supply-and-government-expenditures-must-be-exogenous-variables.-explain-why-this-is-unlikely-to-be-the-case.-as-a-result-what-importance-should-you-attach-to-the-above-results}{%
\subsection{(j) For these coefficients to represent dynamic ultipliers,
the money supply and government expenditures must be exogenous
variables. Explain why this is unlikely to be the case. As a result,
what importance should you attach to the above
results?}\label{j-for-these-coefficients-to-represent-dynamic-ultipliers-the-money-supply-and-government-expenditures-must-be-exogenous-variables.-explain-why-this-is-unlikely-to-be-the-case.-as-a-result-what-importance-should-you-attach-to-the-above-results}}

There is little reason to believe that these government instruments are
exogenous. Even if the monetary base and those components of government
expenditures which do not respond to business cycle fluctuations had
been chosen rather than the above regressors, these instruments respond
to changes in the growth rate of GDP.

In fact, government reaction functions were also estimated at the time
to capture how government instruments respond to changes in target
variables. As a result, the regressors will be correlated with the error
term, OLS estimation is inconsistent, and inference not dependable. It
is hard to imagine how useable information can be retrieved from these
numbers.

\hypertarget{practice-problem-2-stock-watson-empirical-exercise-16.1}{%
\section{Practice Problem 2: Stock-Watson Empirical Exercise
16.1}\label{practice-problem-2-stock-watson-empirical-exercise-16.1}}

Note that the textbook leads you to the data hosted on the official
textbook website. This data does not match what's in the textbook or the
solutions! It doesn't even have the variables you need to answer any of
these subquestions. I had to find the original dataset on an unofficial
website so download it from my recitation folder instead:

\begin{Shaded}
\begin{Highlighting}[]
\NormalTok{macro }\OtherTok{\textless{}{-}} \FunctionTok{read\_excel}\NormalTok{(}\StringTok{\textquotesingle{}data/UsMacro\_Monthly.xlsx\textquotesingle{}}\NormalTok{)}
\FunctionTok{head}\NormalTok{(macro)}
\end{Highlighting}
\end{Shaded}

\begin{verbatim}
## # A tibble: 6 x 6
##    Year Month    IP   Oil   CPI  PCED
##   <dbl> <dbl> <dbl> <dbl> <dbl> <dbl>
## 1  1947     1  13.6    NA  21.5    NA
## 2  1947     2  13.7    NA  21.6    NA
## 3  1947     3  13.7    NA  22      NA
## 4  1947     4  13.6    NA  22      NA
## 5  1947     5  13.7    NA  22.0    NA
## 6  1947     6  13.7    NA  22.1    NA
\end{verbatim}

\hypertarget{part-a-compute-the-monthly-growth-rate-in-ip-expressed-in-percentage-points.-what-are-the-mean-and-standard-deviation-of-ip-growth-over-the-1960m12017m12-sample-period-what-are-the-units-for-ip-growth-percent-percent-per-annum-percent-per-month-or-something-else}{%
\subsection{Part a: Compute the monthly growth rate in IP, expressed in
percentage points. What are the mean and standard deviation of IP growth
over the 1960:M1--2017:M12 sample period? What are the units for IP
growth (percent, percent per annum, percent per month, or something
else)?}\label{part-a-compute-the-monthly-growth-rate-in-ip-expressed-in-percentage-points.-what-are-the-mean-and-standard-deviation-of-ip-growth-over-the-1960m12017m12-sample-period-what-are-the-units-for-ip-growth-percent-percent-per-annum-percent-per-month-or-something-else}}

\begin{Shaded}
\begin{Highlighting}[]
\NormalTok{macro }\SpecialCharTok{\%\textless{}\textgreater{}\%} \FunctionTok{mutate}\NormalTok{(}\AttributeTok{ip.growth =} \DecValTok{100}\SpecialCharTok{*}\NormalTok{(}\FunctionTok{log}\NormalTok{(IP)}\SpecialCharTok{{-}}\FunctionTok{log}\NormalTok{(}\FunctionTok{lag}\NormalTok{(IP))))}

\NormalTok{macro.sample }\OtherTok{\textless{}{-}} \FunctionTok{filter}\NormalTok{(macro, Year }\SpecialCharTok{\textgreater{}=} \DecValTok{1960} \SpecialCharTok{\&}\NormalTok{ Year }\SpecialCharTok{\textless{}=} \DecValTok{2017}\NormalTok{)}
\FunctionTok{mean}\NormalTok{(macro.sample}\SpecialCharTok{$}\NormalTok{ip.growth)}
\end{Highlighting}
\end{Shaded}

\begin{verbatim}
## [1] 0.2235496
\end{verbatim}

\begin{Shaded}
\begin{Highlighting}[]
\FunctionTok{sd}\NormalTok{(macro.sample}\SpecialCharTok{$}\NormalTok{ip.growth)}
\end{Highlighting}
\end{Shaded}

\begin{verbatim}
## [1] 0.7781159
\end{verbatim}

So we have a mean of about 0.22 and a standard deviation of about 0.75.
This is slightly different from the provided solutions, which seem to be
using a sample from the period 1952-2009 for some reason (different
textbook?) rather than 1960-2017.

\hypertarget{part-b-plot-the-value-of-o_t.-why-are-so-many-values-of-o_t-equal-to-0-why-arent-some-values-of-o_t-negative}{%
\subsection{\texorpdfstring{Part b: Plot the value of \(O_t\). Why are
so many values of \(O_t\) equal to 0? Why aren't some values of \(O_t\)
negative?}{Part b: Plot the value of O\_t. Why are so many values of O\_t equal to 0? Why aren't some values of O\_t negative?}}\label{part-b-plot-the-value-of-o_t.-why-are-so-many-values-of-o_t-equal-to-0-why-arent-some-values-of-o_t-negative}}

We want to plot the value of oil. Let's create a Date variable that
combines the info in Year and Month first:

\begin{Shaded}
\begin{Highlighting}[]
\NormalTok{macro }\SpecialCharTok{\%\textless{}\textgreater{}\%} \FunctionTok{mutate}\NormalTok{(}\AttributeTok{Date =} \FunctionTok{as.Date}\NormalTok{(}\FunctionTok{ISOdate}\NormalTok{(Year, Month, }\DecValTok{1}\NormalTok{)))}
\end{Highlighting}
\end{Shaded}

Here, the 1 just tells R to assign it the first day of the month.

Now let's plot the oil time series:

\begin{Shaded}
\begin{Highlighting}[]
\FunctionTok{ggplot}\NormalTok{(macro, }\FunctionTok{aes}\NormalTok{(}\AttributeTok{x =}\NormalTok{ Date, }\AttributeTok{y =}\NormalTok{ Oil)) }\SpecialCharTok{+}
  \FunctionTok{theme\_bw}\NormalTok{() }\SpecialCharTok{+} \CommentTok{\# A black{-}and{-}white theme I like over the gray default}
  \FunctionTok{geom\_line}\NormalTok{() }\SpecialCharTok{+} \CommentTok{\# Draw a line graph using Date on the x and Oil on the y}
  \FunctionTok{xlab}\NormalTok{(}\StringTok{\textquotesingle{}Date\textquotesingle{}}\NormalTok{) }\SpecialCharTok{+} \FunctionTok{ggtitle}\NormalTok{(}\StringTok{\textquotesingle{}Time series of O\textquotesingle{}}\NormalTok{) }\SpecialCharTok{+} \FunctionTok{ylab}\NormalTok{(}\StringTok{\textquotesingle{}\textquotesingle{}}\NormalTok{)}
\end{Highlighting}
\end{Shaded}

\begin{verbatim}
## Warning: Removed 12 rows containing missing values (`geom_line()`).
\end{verbatim}

\includegraphics{08-TimeSeries-DynamicCausalEffects_files/figure-latex/unnamed-chunk-20-1.pdf}

From Exercise 16.1, O here is defined as ``the greater of 0 or the
percentage point difference between oil prices at date t and their
maximum value during the past three years.'' Thus it equals 0 whenever
the price of oil is less than the maximum during the previous three
years.

\hypertarget{part-c-estimate-a-distributed-lag-model-regressing-ip-growth-against-the-current-value-and-18-lagged-values-of-o-including-an-intercept.-what-value-of-the-hac-standard-error-truncation-parameter-did-you-choose-why}{%
\subsection{Part c: Estimate a distributed lag model regressing IP
growth against the current value and 18 lagged values of O, including an
intercept. What value of the HAC standard error truncation parameter did
you choose?
Why?}\label{part-c-estimate-a-distributed-lag-model-regressing-ip-growth-against-the-current-value-and-18-lagged-values-of-o-including-an-intercept.-what-value-of-the-hac-standard-error-truncation-parameter-did-you-choose-why}}

We are estimating a distributed lag model of IP growth onto the current
and 18 lagged values of the variable \(O_t\).

First, let's calculate the HAC truncation parameter m:

\begin{Shaded}
\begin{Highlighting}[]
\NormalTok{m }\OtherTok{\textless{}{-}}  \FloatTok{0.75}\SpecialCharTok{*}\FunctionTok{nrow}\NormalTok{(macro)}\SpecialCharTok{\^{}}\NormalTok{(}\DecValTok{1}\SpecialCharTok{/}\DecValTok{3}\NormalTok{)}
\NormalTok{m }\OtherTok{\textless{}{-}} \FunctionTok{ceiling}\NormalTok{(m)}
\NormalTok{m}
\end{Highlighting}
\end{Shaded}

\begin{verbatim}
## [1] 7
\end{verbatim}

Estimating the requested distributed lag model:

\begin{Shaded}
\begin{Highlighting}[]
\CommentTok{\# Create an id variable like before}
\NormalTok{macro }\SpecialCharTok{\%\textless{}\textgreater{}\%} \FunctionTok{mutate}\NormalTok{(}\AttributeTok{id =} \DecValTok{999}\NormalTok{)}
\NormalTok{oil.dyn }\OtherTok{\textless{}{-}} \FunctionTok{feols}\NormalTok{(ip.growth }\SpecialCharTok{\textasciitilde{}} \FunctionTok{l}\NormalTok{(Oil, }\DecValTok{0}\SpecialCharTok{:}\DecValTok{18}\NormalTok{), macro,}
                   \AttributeTok{panel.id =} \SpecialCharTok{\textasciitilde{}}\NormalTok{ id }\SpecialCharTok{+}\NormalTok{ Date,}
                   \AttributeTok{vcov =} \FunctionTok{NW}\NormalTok{(m))}
\end{Highlighting}
\end{Shaded}

\begin{verbatim}
## NOTE: 30 observations removed because of NA values (LHS: 1, RHS: 30).
\end{verbatim}

\hypertarget{part-d-taken-as-a-group-are-the-coefficients-on-o_t-statistically-significantly-different-from-0}{%
\subsection{\texorpdfstring{Part d: Taken as a group, are the
coefficients on \(O_t\) statistically significantly different from
0?}{Part d: Taken as a group, are the coefficients on O\_t statistically significantly different from 0?}}\label{part-d-taken-as-a-group-are-the-coefficients-on-o_t-statistically-significantly-different-from-0}}

Printing the regression output, including an F-test on all these
regressors:

\begin{Shaded}
\begin{Highlighting}[]
\FunctionTok{etable}\NormalTok{(oil.dyn, }\AttributeTok{fitstat =} \SpecialCharTok{\textasciitilde{}}\NormalTok{ wald }\SpecialCharTok{+}\NormalTok{ wald.p, }\AttributeTok{markdown =}\NormalTok{ T)}
\end{Highlighting}
\end{Shaded}

\begingroup
\centering
\begin{tabular}{lc}
   \tabularnewline \midrule \midrule
   Dependent Variable:           & ip.growth\\  
   Model:                        & (1)\\  
   \midrule
   \emph{Variables}\\
   Constant                      & 0.4275$^{***}$\\   
                                 & (0.0675)\\   
   Oil                           & 0.1553\\   
                                 & (0.8532)\\   
   l(Oil,1)                      & -0.9760\\   
                                 & (0.9559)\\   
   l(Oil,2)                      & -1.403$^{*}$\\   
                                 & (0.7918)\\   
   l(Oil,3)                      & -0.8315\\   
                                 & (0.9295)\\   
   l(Oil,4)                      & -0.4064\\   
                                 & (0.8824)\\   
   l(Oil,5)                      & -0.4202\\   
                                 & (0.7929)\\   
   l(Oil,6)                      & -2.555$^{*}$\\   
                                 & (1.472)\\   
   l(Oil,7)                      & -0.2286\\   
                                 & (0.9797)\\   
   l(Oil,8)                      & 0.8800\\   
                                 & (1.008)\\   
   l(Oil,9)                      & -1.639\\   
                                 & (1.028)\\   
   l(Oil,10)                     & -3.923$^{**}$\\   
                                 & (1.872)\\   
   l(Oil,11)                     & -2.607\\   
                                 & (1.954)\\   
   l(Oil,12)                     & -0.2247\\   
                                 & (1.292)\\   
   l(Oil,13)                     & -1.555\\   
                                 & (1.131)\\   
   l(Oil,14)                     & -1.483\\   
                                 & (0.9120)\\   
   l(Oil,15)                     & -1.479$^{*}$\\   
                                 & (0.8443)\\   
   l(Oil,16)                     & -0.1002\\   
                                 & (0.9037)\\   
   l(Oil,17)                     & 0.4981\\   
                                 & (0.7493)\\   
   l(Oil,18)                     & 0.0096\\   
                                 & (0.9699)\\   
   \midrule
   \emph{Fit statistics}\\
   Wald (joint nullity)          & 1.7399\\  
   Wald (joint nullity), p-value & 0.02607\\  
   \midrule \midrule
   \multicolumn{2}{l}{\emph{Newey-West (L=7) standard-errors in parentheses}}\\
   \multicolumn{2}{l}{\emph{Signif. Codes: ***: 0.01, **: 0.05, *: 0.1}}\\
\end{tabular}
\par\endgroup

Or alternatively:

\begin{Shaded}
\begin{Highlighting}[]
\FunctionTok{wald}\NormalTok{(oil.dyn)}
\end{Highlighting}
\end{Shaded}

\begin{verbatim}
## Wald test, H0: joint nullity of Oil, l(Oil, 1), l(Oil, 2), l(Oil, 3), l(Oil, 4), l(Oil, 5) and 13 others
##  stat = 1.73988, p-value = 0.02607, on 19 and 706 DoF, VCOV: Newey-West (L=7).
\end{verbatim}

With a p-value of 0.026, we can reject the hypothesis of nullity in the
Oil regressors at a 5\% significance level

\hypertarget{part-e-construct-graphs-like-those-in-figure-16.2-showing-the-estimated-dynamic-multipliers-cumulative-multipliers-and-95-confidence-intervals.-comment-on-the-real-world-size-of-the-multipliers.}{%
\subsection{Part e: Construct graphs like those in Figure 16.2, showing
the estimated dynamic multipliers, cumulative multipliers, and 95\%
confidence intervals. Comment on the real-world size of the
multipliers.}\label{part-e-construct-graphs-like-those-in-figure-16.2-showing-the-estimated-dynamic-multipliers-cumulative-multipliers-and-95-confidence-intervals.-comment-on-the-real-world-size-of-the-multipliers.}}

The above model captured dynamic (non-cumulative) effects. To capture
cumulative effects, we would run a related model of up to
\emph{seventeen} lags of \emph{differenced} Oil values and then include
the 18th lag of \emph{non-differenced} Oil. This means we'll need to
create a new variable that is the first difference of the Oil variable:

\begin{Shaded}
\begin{Highlighting}[]
\NormalTok{macro }\SpecialCharTok{\%\textless{}\textgreater{}\%} \FunctionTok{mutate}\NormalTok{(}\AttributeTok{diff.Oil =}\NormalTok{ Oil}\SpecialCharTok{{-}}\FunctionTok{lag}\NormalTok{(Oil))}
\NormalTok{oil.cumdyn }\OtherTok{\textless{}{-}} \FunctionTok{feols}\NormalTok{(ip.growth }\SpecialCharTok{\textasciitilde{}} \FunctionTok{l}\NormalTok{(diff.Oil, }\DecValTok{0}\SpecialCharTok{:}\DecValTok{17}\NormalTok{) }\SpecialCharTok{+} \FunctionTok{l}\NormalTok{(Oil, }\DecValTok{18}\NormalTok{),}
\NormalTok{                        macro,}
                        \AttributeTok{panel.id =} \SpecialCharTok{\textasciitilde{}}\NormalTok{ id }\SpecialCharTok{+}\NormalTok{ Date,}
                        \AttributeTok{vcov =} \FunctionTok{NW}\NormalTok{(m))}
\end{Highlighting}
\end{Shaded}

\begin{verbatim}
## NOTE: 30 observations removed because of NA values (LHS: 1, RHS: 30).
\end{verbatim}

\begin{Shaded}
\begin{Highlighting}[]
\FunctionTok{etable}\NormalTok{(oil.dyn, oil.cumdyn)}
\end{Highlighting}
\end{Shaded}

\begin{verbatim}
##                            oil.dyn         oil.cumdyn
## Dependent Var.:          ip.growth          ip.growth
##                                                      
## Constant        0.4275*** (0.0675) 0.4275*** (0.0675)
## Oil                0.1553 (0.8532)                   
## l(Oil,1)          -0.9760 (0.9559)                   
## l(Oil,2)          -1.403. (0.7918)                   
## l(Oil,3)          -0.8315 (0.9295)                   
## l(Oil,4)          -0.4064 (0.8824)                   
## l(Oil,5)          -0.4202 (0.7929)                   
## l(Oil,6)           -2.555. (1.472)                   
## l(Oil,7)          -0.2286 (0.9797)                   
## l(Oil,8)            0.8800 (1.008)                   
## l(Oil,9)            -1.639 (1.028)                   
## l(Oil,10)          -3.923* (1.872)                   
## l(Oil,11)           -2.607 (1.954)                   
## l(Oil,12)          -0.2247 (1.292)                   
## l(Oil,13)           -1.555 (1.131)                   
## l(Oil,14)          -1.483 (0.9120)                   
## l(Oil,15)         -1.479. (0.8443)                   
## l(Oil,16)         -0.1002 (0.9037)                   
## l(Oil,17)          0.4981 (0.7493)                   
## l(Oil,18)          0.0096 (0.9699)  -18.29*** (4.478)
## diff.Oil                              0.1553 (0.8532)
## l(diff.Oil,1)                         -0.8207 (1.150)
## l(diff.Oil,2)                          -2.224 (1.391)
## l(diff.Oil,3)                         -3.055. (1.786)
## l(diff.Oil,4)                         -3.462. (1.964)
## l(diff.Oil,5)                         -3.882. (2.298)
## l(diff.Oil,6)                         -6.437* (2.884)
## l(diff.Oil,7)                         -6.666* (2.813)
## l(diff.Oil,8)                         -5.786* (2.708)
## l(diff.Oil,9)                         -7.425* (2.903)
## l(diff.Oil,10)                       -11.35** (3.579)
## l(diff.Oil,11)                      -13.96*** (3.917)
## l(diff.Oil,12)                      -14.18*** (4.259)
## l(diff.Oil,13)                      -15.74*** (4.472)
## l(diff.Oil,14)                      -17.22*** (4.682)
## l(diff.Oil,15)                      -18.70*** (4.653)
## l(diff.Oil,16)                      -18.80*** (4.493)
## l(diff.Oil,17)                      -18.30*** (4.441)
## _______________ __________________ __________________
## S.E. type         Newey-West (L=7)   Newey-West (L=7)
## Observations                   726                726
## R2                         0.07860            0.07860
## Adj. R2                    0.05380            0.05380
## ---
## Signif. codes: 0 '***' 0.001 '**' 0.01 '*' 0.05 '.' 0.1 ' ' 1
\end{verbatim}

Then it is easy to draw the desired graphs using fixest's coefplot
function:

\begin{Shaded}
\begin{Highlighting}[]
\FunctionTok{coefplot}\NormalTok{(oil.dyn, }\AttributeTok{drop =} \StringTok{\textquotesingle{}Constant\textquotesingle{}}\NormalTok{,}
         \AttributeTok{main =} \StringTok{\textquotesingle{}Dynamic multipliers\textquotesingle{}}\NormalTok{, }\AttributeTok{xlab =} \StringTok{\textquotesingle{}Lag\textquotesingle{}}\NormalTok{)}
\end{Highlighting}
\end{Shaded}

\includegraphics{08-TimeSeries-DynamicCausalEffects_files/figure-latex/unnamed-chunk-26-1.pdf}

\begin{Shaded}
\begin{Highlighting}[]
\FunctionTok{coefplot}\NormalTok{(oil.cumdyn, }\AttributeTok{drop =} \FunctionTok{c}\NormalTok{(}\StringTok{\textquotesingle{}Constant\textquotesingle{}}\NormalTok{, }\StringTok{\textquotesingle{}lag(Oil, 18)\textquotesingle{}}\NormalTok{),}
         \AttributeTok{main =} \StringTok{\textquotesingle{}Cumulative multipliers\textquotesingle{}}\NormalTok{, }\AttributeTok{xlab =} \StringTok{\textquotesingle{}Lag\textquotesingle{}}\NormalTok{)}
\end{Highlighting}
\end{Shaded}

\includegraphics{08-TimeSeries-DynamicCausalEffects_files/figure-latex/unnamed-chunk-26-2.pdf}

Or with some customization:

\begin{Shaded}
\begin{Highlighting}[]
\NormalTok{dyn }\OtherTok{\textless{}{-}} \FunctionTok{coefplot}\NormalTok{(oil.dyn, }\AttributeTok{drop =} \StringTok{\textquotesingle{}Constant\textquotesingle{}}\NormalTok{,}
         \AttributeTok{main =} \StringTok{\textquotesingle{}The effect of oil prices on IP growth rate}\SpecialCharTok{\textbackslash{}n}\StringTok{Dynamic multipliers\textquotesingle{}}\NormalTok{,}
         \AttributeTok{xlab =} \StringTok{\textquotesingle{}Lag\textquotesingle{}}\NormalTok{,}
         \AttributeTok{ci.join =}\NormalTok{ T, }\AttributeTok{ci.lty =} \DecValTok{0}\NormalTok{, }\AttributeTok{ci.fill =}\NormalTok{ T, }\AttributeTok{ci.fill.par =} \FunctionTok{list}\NormalTok{(}\AttributeTok{col =} \StringTok{\textquotesingle{}lightblue\textquotesingle{}}\NormalTok{),}
         \AttributeTok{pt.join =}\NormalTok{ T)}
\end{Highlighting}
\end{Shaded}

\includegraphics{08-TimeSeries-DynamicCausalEffects_files/figure-latex/unnamed-chunk-27-1.pdf}

\begin{Shaded}
\begin{Highlighting}[]
\NormalTok{cumdyn }\OtherTok{\textless{}{-}} \FunctionTok{coefplot}\NormalTok{(oil.cumdyn, }\AttributeTok{drop =} \StringTok{\textquotesingle{}Constant\textquotesingle{}}\NormalTok{, }\AttributeTok{ci.join =}\NormalTok{ T,}
         \AttributeTok{main =} \StringTok{\textquotesingle{}The effect of oil prices on IP growth rate}\SpecialCharTok{\textbackslash{}n}\StringTok{Cumulative dynamic multipliers\textquotesingle{}}\NormalTok{,}
         \AttributeTok{xlab =} \StringTok{\textquotesingle{}Lag\textquotesingle{}}\NormalTok{,}
         \AttributeTok{ci.lty =} \DecValTok{0}\NormalTok{, }\AttributeTok{ci.fill =}\NormalTok{ T, }\AttributeTok{ci.fill.par =} \FunctionTok{list}\NormalTok{(}\AttributeTok{col =} \StringTok{\textquotesingle{}red\textquotesingle{}}\NormalTok{),}
         \AttributeTok{pt.join =}\NormalTok{ T)}
\end{Highlighting}
\end{Shaded}

\includegraphics{08-TimeSeries-DynamicCausalEffects_files/figure-latex/unnamed-chunk-27-2.pdf}

\hypertarget{practice-problems-3-5-stock-watson-non-empirical-exercises-14.1-14.2-14.5-on-ipad}{%
\section{Practice Problems 3-5: Stock-Watson non-empirical exercises
14.1, 14.2, 14.5 on
iPad}\label{practice-problems-3-5-stock-watson-non-empirical-exercises-14.1-14.2-14.5-on-ipad}}

\end{document}
